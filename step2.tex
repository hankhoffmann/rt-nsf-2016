In step 2, we will develop new algorithms and techniques that address
energy consumption for mixed critical real-time systems such as
automotive computing.  In both cases, we plan to leverage two key
insights about heterogeneous architectures like those presented in
Figure~\ref{tradeoffs}:
\begin{itemize}
\item Their energy efficiency increases as they slow down.
\item They have tremendous additional compute capacity that can only
  be used for short amounts of time.
\end{itemize}

Specifically, we will develop two techniques, each based off one of
these insights, and each providing a different capability:
\begin{itemize}
\item \textbf{Pace-to-sprint}: The goal of this technique is to reduce
  energy consumption for hard real-time workloads.  Where traditional
  techniques allocate for worst case and then idle if work is
  completed early (i.e., race-to-idle), pace-to-sprint will develop
  scheduling techniques to spend as much time as possible in slower,
  more energy-efficient states.
\item \textbf{Energy Guarantees}: The goal of this technique will be
  to guarantee energy consumption for mixed critical workloads; i.e.,
  workloads that contain jobs with both hard and soft real-time
  requirements.  In this work, we will use the additional,
  unsustainable compute capacity to process hard real-time tasks and
  ensure that their deadlines are respected (building off our work in
  step 1).  In addition, we will develop algorithms and interfaces for
  admitting additional soft real-time jobs that will only be scheduled
  if they do not violate energy constraints.  Thus, this technique
  will allow us to bound both timeliness and energy consumption,
  possibly at the cost of not executing some non-critical tasks.
\end{itemize}


\subsubsection{Pace-to-Sprint}
We will develop pace-to-sprint techniques to minimize energy
consumption.  The key insight here is that allocating for worst case
execution time...

Linear optimization?




\subsubsection{Energy Guarantees for Mixed Critical Workloads}
We will develop techniques to provide energy consumption guarantees
for systems with both hard and soft real-time requirements.

Take energy budget.  Divide into periods.  Determine average power per
scheduling period.

Hard real-time tasks scheduled using big cores.  When those complete,
determine remaining available power.  Run best-effort tasks in
configuration that meets available power requirement.


\subsubsection{Evaluation}

We will evaluate our techniques along two metrics: energy consumption
and schedulability.

Energy consumption of pace-to-sprint should be lower than existing
techniques, specific savings will depend on workload, but maybe a
factor of two better than race.

Schedulability of our guaranteed techniques should be higher than
current techniques.
