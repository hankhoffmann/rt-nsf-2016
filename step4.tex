In order to further access whether our proposed methods can be applied in practical systems, we intend to conduct case-study evaluations using actual workloads from an advanced automotive system. Due to space constraints, we describe one proposed case study in (some) detail, and briefly note others.

\textit{Automotive workloads.} As mentioned earlier, a strong motivation behind the proposed project is to leverage heterogeneous computing platforms to support a set of advanced driving-assisted tools in an power-constrained automotive system. In prior work on such systems, timing predictability and resource provisioning issues have been the major focus while the critical energy efficiency issues are ignored. Due to lacking energy-efficient real-time scheduling schemes, significant energy lost may occur to guarantee timing. Even worse, without analytically verifiable energy and timing  guarantees, such driving-assisted features could not easily be legally certifiable, and safety thus becomes a major issue. 

%\begin{wrapfigure}{R}{0.5\textwidth}
%\centering
%\vspace{0mm}
%\includegraphics[width=.48\columnwidth]{images/breakdown.pdf}
%\caption{Workload breakdown of a typical real-time object recognition task.}
%\vspace{-1mm}
%\label{fig:breakdown}
%\end{wrapfigure}

We plan to evaluate a set of common driving-assisted workloads seen in automotive systems using one or more preferable scheduling plugins identified in Section~\ref{sec:step3}. Real-time object recognition represents one of such workloads, which serves as a major task in implementing the automatic collision avoidance and the traffic sign recognition functionality. %Fig.~\ref{fig:breakdown} shows the workload breakdown and parameters of a typical object recognition task using a popular recognition algorithm~\cite{?}.   
Each camera sensor installed on a vehicle continuously captures images at a certain frequency. Then each image will be compressed to reduce irrelevance and redundancy of the image data and processed or stored in an efficient manner. Each compressed image will then be processed, where computation-intensive object recognition and analysis algorithms will be executed to recognize the desired objects in each image. Performing such tasks in an automotive system is challenging because different stages often require different amount of computational capacity and energy.  

We plan to use our proposed energy efficient real-time scheduling methods to support such tasks and evaluate the performance w.r.t. four important metrics: \textit{(i)} timing performance, i.e., schedulability and max/average response times, \textit{(ii)} energy performance, i.e., max/average power and energy consumption, \textit{(iii}) overall system throughput, i.e., number of object recognition tasks that can be supported in parallel, and \textit{(iv)} recognition accuracy.  We plan to adopt a number of existing image compression and object recognition algorithms with different runtime complexity to investigate the trade-offs among energy consumption, response time performance, and recognition accuracy.  
 %Moreover, given specific timing requirements (enforced by safety requirements) such as response time bounds, our measured system performance may be useful in determining minimum system provisioning requirements. 
  In addition to the real-time object recognition task, there exist many other computer vision-based workloads such as autonomous navigation, automatic lane following, and intelligent cruise control. We will study a mixed set of such workloads in the case study. Among these workloads seen in an advanced automotive system, video and image processing-based tasks are the most common ones, as briefly noted below.

\textit{Real-time video decoding.} Emerging demands for real-time video decoding brings the current hardware resources to its limits. We would like to investigate whether our proposed scheduling methods can efficiently utilize heterogeneous processor capacity and thus support the concurrent processing of more videos with guaranteed timing correctness while minimizing energy consumption. %As discussed in Section~\cite{sec:hardware}, for many video decoding workloads, executing them on big cores with the lowest DVFS setting may still yields considerably higher energy consumption than on LITTLE cores with the highest DVFS setting. 
 We will also investigate the tradeoffs among video quality, responsiveness, and energy consumption. An experiment we intend to conduct is to compare our methods against the native schedulers on big.LITTLE, which could answer an interesting question: whether analysis-based energy-efficient scheduling techniques are superior to widely used general-purpose OS schedulers in terms of energy and timing performance?

\textit{Real-time image analysis.} Image processing is also a candidate that may benefit greatly from using heterogeneous computing platforms. One specific application we plan to evaluate is the high dynamic range imaging (HDRI), which is used to capture a greater dynamic range between the lightest and darkest areas of an image. HDRI can be naturally modeled as a set of recurrent real-time tasks that exhibit various levels of computational intensiveness and energy characteristics  (refer to \cite{kuang2007evaluating} for a detailed description of HDRI). Our work may include the development of systems that can process a large-scale set of images in a timing-correct fashion while dramatically minimizing energy consumption by consider energy heterogeneity inherent to both workloads and processors.