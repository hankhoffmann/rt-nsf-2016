%\section{Broader Impact of Proposed Research}
%\label{sec:broader}

\noindent {\bf Scientific and societal impact of research:} 
The results from this project will be of interest to industry as well
as US government agencies.  Developed techniques will synergistically
improve safety critical systems by 1) allowing them to incorporate
more functionality in the same platform (reducing costs and adding new
capabilities) or 2) greatly reducing the energy consumption of current
systems (allowing extended time between charging and allowing
harvested energy systems to perform more when energy resources are
scarce.  Increasing our ability to build safey-critical,
energy-efficient systems will benefit medical, defense,
transportation, and automotive applications.  

%\vspace{0.1cm}
\noindent {\bf Dissemination of of research and collaboration:} 
We will present tutorials and workshops at conferences, and seek to
publish our outcomes in premier embedded, real-time, and operating
system journals and conferences, e.g., SOSP, ASPLOS, RTSS.  We will
make all the real-time benchmarks, simulation framework/models,
software, and tools publicly available, so that others can repeat
experiments, and adopt and build on the ideas.  PI Liu and his
collaborators have already <CONG: list any publicly available software
or industry impacts here>.  PI Hoffmann and his collaborators have
released a number of software packages as open source.  This includes
the Application Heartbeats API for monitoring software performance and
health \cite{heartbeats,heartbeatsweb}, the POET toolkit for portable
energy efficiency \cite{POET,poetweb}, and LEO for learning computer
system performance and energy tradeoffs \cite{LEO,leoweb}.

The PIs also have a strong track record of transitioning research and
technology developed in academia into industry.  PI Liu <CONG: fill in
any industry collaborations> PI Hoffmann was part of the team that
transferred MIT's Raw Architecture to industry through the founding of
the Tilera Corporation~\cite{RSP,TILE}.  At MIT Lincoln Laboratory, PI
Hoffmann developed the foundations for the VSIPL++ parallel image
processing standard which is now used worldwide~\cite{VSIPL++}.  In
collaboration with Mozilla Reseach, PI Hoffmann's students have
incorporated their dynamic instrumentation and energy monitoring
framework into the open source release of Servo, Mozilla's open-source
mobile web browser \cite{}.


%\vspace{0.1cm}
\noindent {\bf Outreach and Diversity:}
We will seek to actively recruit under-represented minority and women
graduate and undergraduate students into our research groups.  PI
Hoffmann is actively involved in encouraging under-represented
minorities to participate in research through the Leadership Alliance
at the University of Chicago.  Currently 33\% of his PhD students are
from traditionally under-represented backgrounds.  The Leadership
Alliance is a national academic consortium of 33 colleges and
universities focused on training and mentoring a diverse group of
undergraduate students into competitive graduate programs and
professional research careers.  The University of Chicago is among the
22 Alliance institutions that offer the Summer Research Early
Identification Program (SR-EIP). In summer 2015, PI Hoffmann mentored
two students through the Leadership Alliance program
\cite{LA-article}.
%The University of Chicago SR-EIP provides a 9-week research internship.


\vspace{0.1cm}
\noindent {\bf Curricular Development Activities:}
PI Liu teaches... <CONG: fill in>

PI Hoffmann teaches a graduate class on power and energy-aware
computing.  This course covers power and energy issues at the
operating system, architecture, and microarchitectural levels. The
proposed research will both help him incorporate new lessons on energy
management into the class and it will provide research problems for
students' course projects.  This class is currently being taught for
the second time. The first iteration of the class produced 2 projects
that are in submission for publication and 1 project that became the
student's Master's thesis.
